\section{Descifrado de criptogramas}

\subsection{Sexto mensaje - Datos}

Se nos proporciona la siguiente información:

\begin{itemize}
\item $n = 1903$
\item $p = 173$
\item $q = 11$
\item $d = 71$
\item $z = (p-1)(q-1) = (173-1)(11-1) = 1720$
\end{itemize}

Y el siguiente mensaje cifrado:

\begin{verbatim}
1497 1583 791 1497 1394
791 1583 123 937 786
1583
\end{verbatim}

\subsection{Cálculo del inverso multiplicativo módulo $z$}

Necesitamos encontrar $e$ tal que $e \cdot d \equiv 1 \pmod{z}$, es decir, el inverso multiplicativo de $d$ módulo $z$.

Para ello utilizaremos el algoritmo de Euclides extendido:

\textbf{Inicialización del algoritmo de Euclides extendido:}

Estamos buscando $e$ tal que $e \cdot 71 \equiv 1 \pmod{1720}$

Inicializamos las variables según el algoritmo:

$g_0 = 1720$, $g_1 = 71$

$u_0 = 1$, $u_1 = 0$

$v_0 = 0$, $v_1 = 1$



\textbf{Iteración 1:}

Calculamos $y_2 = \lfloor g_0 / g_i \rfloor = \lfloor 1720 / 71 \rfloor = 24$

Calculamos $g_2 = g_0 - y_2 \cdot g_i = 1720 - 24 \cdot 71 = 16$

Calculamos $u_2 = u_0 - y_2 \cdot u_i = 1 - 24 \cdot 0 = 1$

Calculamos $v_2 = v_0 - y_2 \cdot v_i = 0 - 24 \cdot 1 = -24$



\textbf{Iteración 2:}

Calculamos $y_3 = \lfloor g_1 / g_i \rfloor = \lfloor 71 / 16 \rfloor = 4$

Calculamos $g_3 = g_1 - y_3 \cdot g_i = 71 - 4 \cdot 16 = 7$

Calculamos $u_3 = u_1 - y_3 \cdot u_i = 0 - 4 \cdot 1 = -4$

Calculamos $v_3 = v_1 - y_3 \cdot v_i = 1 - 4 \cdot -24 = 97$



\textbf{Iteración 3:}

Calculamos $y_4 = \lfloor g_2 / g_i \rfloor = \lfloor 16 / 7 \rfloor = 2$

Calculamos $g_4 = g_2 - y_4 \cdot g_i = 16 - 2 \cdot 7 = 2$

Calculamos $u_4 = u_2 - y_4 \cdot u_i = 1 - 2 \cdot -4 = 9$

Calculamos $v_4 = v_2 - y_4 \cdot v_i = -24 - 2 \cdot 97 = -218$



\textbf{Iteración 4:}

Calculamos $y_5 = \lfloor g_3 / g_i \rfloor = \lfloor 7 / 2 \rfloor = 3$

Calculamos $g_5 = g_3 - y_5 \cdot g_i = 7 - 3 \cdot 2 = 1$

Calculamos $u_5 = u_3 - y_5 \cdot u_i = -4 - 3 \cdot 9 = -31$

Calculamos $v_5 = v_3 - y_5 \cdot v_i = 97 - 3 \cdot -218 = 751$



\textbf{Iteración 5:}

Calculamos $y_6 = \lfloor g_4 / g_i \rfloor = \lfloor 2 / 1 \rfloor = 2$

Calculamos $g_6 = g_4 - y_6 \cdot g_i = 2 - 2 \cdot 1 = 0$

Calculamos $u_6 = u_4 - y_6 \cdot u_i = 9 - 2 \cdot -31 = 71$

Calculamos $v_6 = v_4 - y_6 \cdot v_i = -218 - 2 \cdot 751 = -1720$



\textbf{Resultado final:}

El algoritmo ha terminado porque $g_5 = 0$

El inverso se encuentra en el valor de $v$ en el penúltimo paso: $v_4 = 751$

Como $v_4 \geq 0$, no es necesario ajustar el valor.

Por lo tanto, el inverso multiplicativo de 71 módulo 1720 es 751

Verificación: $71 \cdot 751 \equiv 1 \pmod{1720}$

\begin{table}[h]
\centering
\begin{tabular}{|c|c|c|c|c|}
\hline
$i$ & $y_i$ & $g_i$ & $u_i$ & $v_i$ \\ \hline
0 & - & 1720 & 1 & 0 \\ \hline
1 & - & 71 & 0 & 1 \\ \hline
2 & 24 & 16 & 1 & -24 \\ \hline
3 & 4 & 7 & -4 & 97 \\ \hline
4 & 2 & 2 & 9 & -218 \\ \hline
5 & 3 & 1 & -31 & 751 \\ \hline
6 & 2 & 0 & 71 & -1720 \\ \hline
\end{tabular}
\caption{Cálculo del inverso modular utilizando el algoritmo de Euclides extendido}
\label{tab:euclides}
\end{table}

Por lo tanto, $e = 751$.

\subsection{Descifrado del mensaje}

Para descifrar el mensaje, utilizamos la fórmula $M = C^e \bmod n$, donde $e = 751$ y $n = 1903$.

\subsubsection{Proceso de descifrado}

$C = 1497$: $M = 1497^{751} \bmod 1903 = 67$ (ASCII: 'C')

$C = 1583$: $M = 1583^{751} \bmod 1903 = 65$ (ASCII: 'A')

$C = 791$: $M = 791^{751} \bmod 1903 = 76$ (ASCII: 'L')

$C = 1497$: $M = 1497^{751} \bmod 1903 = 67$ (ASCII: 'C')

$C = 1394$: $M = 1394^{751} \bmod 1903 = 85$ (ASCII: 'U')

$C = 791$: $M = 791^{751} \bmod 1903 = 76$ (ASCII: 'L')

$C = 1583$: $M = 1583^{751} \bmod 1903 = 65$ (ASCII: 'A')

$C = 123$: $M = 123^{751} \bmod 1903 = 68$ (ASCII: 'D')

$C = 937$: $M = 937^{751} \bmod 1903 = 79$ (ASCII: 'O')

$C = 786$: $M = 786^{751} \bmod 1903 = 82$ (ASCII: 'R')

$C = 1583$: $M = 1583^{751} \bmod 1903 = 65$ (ASCII: 'A')

\begin{table}[h]
\centering
\begin{tabular}{|c|c|c|}
\hline
$C$ & $M$ & ASCII \\ \hline
1497 & 67 & 'C' \\ \hline
1583 & 65 & 'A' \\ \hline
791 & 76 & 'L' \\ \hline
1497 & 67 & 'C' \\ \hline
1394 & 85 & 'U' \\ \hline
791 & 76 & 'L' \\ \hline
1583 & 65 & 'A' \\ \hline
123 & 68 & 'D' \\ \hline
937 & 79 & 'O' \\ \hline
786 & 82 & 'R' \\ \hline
1583 & 65 & 'A' \\ \hline
\end{tabular}
\caption{Descifrado RSA: $m = c^e \bmod n$}
\label{tab:descifrado}
\end{table}
\subsubsection{Mensaje descifrado}

El mensaje descifrado es:

\begin{verbatim}
CALCULADORA
\end{verbatim}






\subsection{Séptimo mensaje - Datos}

Se nos proporciona la siguiente información:

\begin{itemize}
\item $n = 2581$
\item $p = 29$
\item $q = 89$
\item $d = 5$
\item $z = (p-1)(q-1) = (29-1)(89-1) = 2464$
\end{itemize}

Y el siguiente mensaje cifrado:

\begin{verbatim}
1236 2131 15 2131 202 2069 1035 2069 1104 662
\end{verbatim}

\subsection{Cálculo del inverso multiplicativo módulo $z$}

Necesitamos encontrar $e$ tal que $e \cdot d \equiv 1 \pmod{z}$, es decir, el inverso multiplicativo de $d$ módulo $z$. Para ello utilizaremos el algoritmo de Euclides extendido:
\\\\
\textbf{Inicialización del algoritmo de Euclides extendido:}

Estamos buscando $e$ tal que $e \cdot 5 \equiv 1 \pmod{2464}$

Inicializamos las variables según el algoritmo:

$g_0 = 2464$, $g_1 = 5$
$u_0 = 1$, $u_1 = 0$
$v_0 = 0$, $v_1 = 1$


\textbf{Iteración 1:}

Calculamos $y_2 = \lfloor g_0 / g_i \rfloor = \lfloor 2464 / 5 \rfloor = 492$

Calculamos $g_2 = g_0 - y_2 \cdot g_i = 2464 - 492 \cdot 5 = 4$

Calculamos $u_2 = u_0 - y_2 \cdot u_i = 1 - 492 \cdot 0 = 1$

Calculamos $v_2 = v_0 - y_2 \cdot v_i = 0 - 492 \cdot 1 = -492$



\textbf{Iteración 2:}

Calculamos $y_3 = \lfloor g_1 / g_i \rfloor = \lfloor 5 / 4 \rfloor = 1$

Calculamos $g_3 = g_1 - y_3 \cdot g_i = 5 - 1 \cdot 4 = 1$

Calculamos $u_3 = u_1 - y_3 \cdot u_i = 0 - 1 \cdot 1 = -1$

Calculamos $v_3 = v_1 - y_3 \cdot v_i = 1 - 1 \cdot -492 = 493$



\textbf{Iteración 3:}

Calculamos $y_4 = \lfloor g_2 / g_i \rfloor = \lfloor 4 / 1 \rfloor = 4$

Calculamos $g_4 = g_2 - y_4 \cdot g_i = 4 - 4 \cdot 1 = 0$

Calculamos $u_4 = u_2 - y_4 \cdot u_i = 1 - 4 \cdot -1 = 5$

Calculamos $v_4 = v_2 - y_4 \cdot v_i = -492 - 4 \cdot 493 = -2464$



\textbf{Resultado final:}

El algoritmo ha terminado porque $g_3 = 0$

El inverso se encuentra en el valor de $v$ en el penúltimo paso: $v_2 = 493$

Como $v_2 \geq 0$, no es necesario ajustar el valor.

Por lo tanto, el inverso multiplicativo de 5 módulo 2464 es 493

Verificación: $5 \cdot 493 \equiv 1 \pmod{2464}$

\begin{table}[h]
\centering
\begin{tabular}{|c|c|c|c|c|}
\hline
$i$ & $y_i$ & $g_i$ & $u_i$ & $v_i$ \\ \hline
0 & - & 2464 & 1 & 0 \\ \hline
1 & - & 5 & 0 & 1 \\ \hline
2 & 492 & 4 & 1 & -492 \\ \hline
3 & 1 & 1 & -1 & 493 \\ \hline
4 & 4 & 0 & 5 & -2464 \\ \hline
\end{tabular}
\caption{Cálculo del inverso modular utilizando el algoritmo de Euclides extendido}
\label{tab:euclides}
\end{table}

Por lo tanto, $e = 493$.

\subsection{Descifrado del mensaje}

Para descifrar el mensaje, utilizamos la fórmula $M = C^e \bmod n$, donde $e = 493$ y $n = 2581$.

\subsubsection{Proceso de descifrado}

$C = 1236$: $M = 1236^{493} \bmod 2581 = 84$ (ASCII: 'T')

$C = 2131$: $M = 2131^{493} \bmod 2581 = 69$ (ASCII: 'E')

$C = 15$: $M = 15^{493} \bmod 2581 = 76$ (ASCII: 'L')

$C = 2131$: $M = 2131^{493} \bmod 2581 = 69$ (ASCII: 'E')

$C = 202$: $M = 202^{493} \bmod 2581 = 86$ (ASCII: 'V')

$C = 2069$: $M = 2069^{493} \bmod 2581 = 73$ (ASCII: 'I')

$C = 1035$: $M = 1035^{493} \bmod 2581 = 83$ (ASCII: 'S')

$C = 2069$: $M = 2069^{493} \bmod 2581 = 73$ (ASCII: 'I')

$C = 1104$: $M = 1104^{493} \bmod 2581 = 79$ (ASCII: 'O')

$C = 662$: $M = 662^{493} \bmod 2581 = 78$ (ASCII: 'N')

\begin{table}[h]
\centering
\begin{tabular}{|c|c|c|}
\hline
$C$ & $M$ & ASCII \\ \hline
1236 & 84 & 'T' \\ \hline
2131 & 69 & 'E' \\ \hline
15 & 76 & 'L' \\ \hline
2131 & 69 & 'E' \\ \hline
202 & 86 & 'V' \\ \hline
2069 & 73 & 'I' \\ \hline
1035 & 83 & 'S' \\ \hline
2069 & 73 & 'I' \\ \hline
1104 & 79 & 'O' \\ \hline
662 & 78 & 'N' \\ \hline
\end{tabular}
\caption{Descifrado RSA: $m = c^e \bmod n$}
\label{tab:descifrado}
\end{table}
\subsubsection{Mensaje descifrado}

El mensaje descifrado es:

\begin{verbatim}
TELEVISION
\end{verbatim}