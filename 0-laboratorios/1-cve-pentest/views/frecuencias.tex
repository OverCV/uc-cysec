\section{Criptoanálisis mediante Análisis de Frecuencias}

El análisis de frecuencias es una técnica fundamental en criptoanálisis, especialmente para cifrados de sustitución. Se basa en el principio de que cada lenguaje tiene patrones estadísticos característicos en la frecuencia de aparición de letras y grupos de letras.

\begin{figure}[ht]
	\centering
	\begin{tikzpicture}
		\begin{axis}[
				ybar,
				symbolic x coords={a,e,o,i,s,n,r,l,d,t,c,u,m,p,b,g,v,y,q,h,f,z,j,ñ,x,k,w},
				xtick=data,
				ylabel={Frecuencia (\%)},
				ymin=0,
				ymax=14,
				bar width=6pt,
				enlarge x limits=0.02,
				nodes near coords align={vertical},
				every node near coord/.append style={font=\tiny},
				width=\textwidth,
				height=7cm
			]
			\addplot[fill=infoblue] coordinates {
					(a,12.53) (e,13.68) (o,8.68) (i,6.25) (s,7.88) (n,7.01) (r,6.87) (l,4.97)
					(d,5.86) (t,4.63) (c,4.68) (u,3.93) (m,3.15) (p,2.51) (b,1.42) (g,1.01)
					(v,0.90) (y,0.90) (q,0.88) (h,0.70) (f,0.69) (z,0.52) (j,0.44) (ñ,0.31)
					(x,0.22) (k,0.01) (w,0.01)
				};
		\end{axis}
	\end{tikzpicture}
	\caption{Frecuencia de aparición de letras en español}
	\label{fig:frecuencias-espanol}
\end{figure}

\subsection{Fundamentos del Análisis de Frecuencias}

\begin{cryptomethod}{Técnica de Análisis de Frecuencias}
	El proceso básico de análisis de frecuencias incluye:

	\begin{enumerate}
		\item \textbf{Cálculo de Frecuencias:} Conteo de apariciones de cada símbolo en el texto cifrado.

		\item \textbf{Comparación con Patrones:} Contrastar con las frecuencias conocidas del idioma objetivo.

		\item \textbf{Asignación Tentativa:} Asociar símbolos cifrados con letras del alfabeto según su frecuencia.

		\item \textbf{Análisis de N-gramas:} Examinar frecuencias de pares o tríos de letras (bigramas, trigramas).

		\item \textbf{Refinamiento:} Ajustar la asignación basándose en patrones lingüísticos y prueba-error.
	\end{enumerate}
\end{cryptomethod}

\subsection{Implementación Práctica}

Para demostrar el análisis de frecuencias, desarrollamos un script en Python que analiza un texto cifrado:

\begin{lstlisting}[language=Python, caption={Implementación de Análisis de Frecuencias}, label=lst:analisis-frecuencias]
def analisis_frecuencias(texto_cifrado):
    # Eliminar espacios y convertir a minúsculas
    texto = texto_cifrado.lower().replace(" ", "")
    
    # Contar frecuencias individuales
    frecuencias = {}
    for caracter in texto:
        if caracter in frecuencias:
            frecuencias[caracter] += 1
        else:
            frecuencias[caracter] = 1
    
    # Calcular porcentajes
    total = len(texto)
    for caracter in frecuencias:
        frecuencias[caracter] = (frecuencias[caracter] / total) * 100
    
    # Ordenar por frecuencia descendente
    ordenado = sorted(frecuencias.items(), key=lambda x: x[1], reverse=True)
    
    # Contar bigramas
    bigramas = {}
    for i in range(len(texto) - 1):
        bigrama = texto[i:i+2]
        if bigrama in bigramas:
            bigramas[bigrama] += 1
        else:
            bigramas[bigrama] = 1
    
    # Ordenar bigramas por frecuencia
    bigramas_ordenados = sorted(bigramas.items(), key=lambda x: x[1], reverse=True)
    
    return {
        'caracteres': ordenado[:10],  # Top 10 caracteres
        'bigramas': bigramas_ordenados[:10]  # Top 10 bigramas
    }

# Ejemplo de uso
texto_cifrado = "l % d s k ( c )"
resultado = analisis_frecuencias(texto_cifrado)
print("Caracteres más frecuentes:", resultado['caracteres'])
print("Bigramas más frecuentes:", resultado['bigramas'])
\end{lstlisting}

\subsection{Caso de Estudio: Descifrado de Texto}

\begin{cryptoanalysis}{Análisis del Texto Cifrado}
	Aplicamos nuestra herramienta al texto cifrado proporcionado:
	\begin{verbatim}
l % d s k ( c )
\end{verbatim}

	El análisis revela:
	\begin{itemize}
		\item Caracteres más frecuentes:
		      ' ' (25.0\%), 'l' (12.5\%), '\%' (12.5\%), 'd' (12.5\%), 's' (12.5\%)

		\item Por la estructura y símbolos empleados, parece tratarse de un cifrado por sustitución simple donde cada carácter del texto original ha sido reemplazado por un símbolo específico.

		\item La brevedad del texto limita la efectividad del análisis estadístico, ya que el tamaño de la muestra no es suficiente para establecer patrones confiables.
	\end{itemize}

	Aunque las frecuencias observadas no coinciden exactamente con las esperadas en español, podemos hacer algunas inferencias:
	\begin{itemize}
		\item El carácter 'l' (12.5%) podría corresponder a 'e' o 'a', letras más frecuentes en español.

		\item El número de símbolos distintos (7) sugiere un texto muy corto o un cifrado que utiliza un alfabeto reducido.
	\end{itemize}
\end{cryptoanalysis}