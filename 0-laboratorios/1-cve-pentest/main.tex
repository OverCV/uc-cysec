\documentclass[12pt,a4paper]{article}

% Paquetes esenciales para artículos de Ciberseguridad
\usepackage{cite}
\usepackage{amsmath,amssymb,amsfonts}
\usepackage{algorithm}
\usepackage{algpseudocode}
\usepackage{graphicx}
\usepackage{textcomp}
\usepackage{xcolor}
\usepackage{booktabs}
\usepackage{array}
\usepackage{multirow}
\usepackage{siunitx}
\usepackage{listings}

\usepackage{placeins}

\usepackage{tikz}
\usetikzlibrary{shapes,arrows,positioning,fit,backgrounds,calc}
\usepackage{pgfplots}

\pgfplotsset{compat=1.16}

\usepackage{hyperref}
\usepackage{tabularx}
\usepackage{enumitem}
\usepackage{mdframed}
\usepackage[left=2.5cm,right=2.5cm,top=2.5cm,bottom=2.5cm]{geometry}
\usepackage{authblk} % Para gestionar autores y afiliaciones de manera sencilla

\usepackage[spanish]{babel} % idioma en español


% Definición de colores para ciberseguridad
\definecolor{alertred}{RGB}{235,54,42}
\definecolor{warningyellow}{RGB}{250,190,40}
\definecolor{securegreen}{RGB}{46,204,113}
\definecolor{infoblue}{RGB}{52,152,219}
\definecolor{vulnerablepurple}{RGB}{155,89,182}
\definecolor{codegray}{RGB}{240,240,240}
\definecolor{commentgreen}{RGB}{63,127,95}
\definecolor{codeblue}{RGB}{0,0,255}
\definecolor{codered}{RGB}{176,0,0}
\definecolor{cryptogreen}{RGB}{0,128,0}
\definecolor{encryptionbg}{RGB}{245,245,245}

% Configuración para listings con soporte mejorado para bash
\lstdefinelanguage{Bash}{
  keywords={if, then, else, elif, fi, for, do, done, while, until, case, esac, function},
  keywordstyle=\color{blue},
  sensitive=true,
  comment=[l]{\#},
  commentstyle=\color{commentgreen},
  stringstyle=\color{red!70!black},
  morestring=[b]",
  morestring=[b]'
}

% Definición para código fuente genérico incluyendo lenguajes de programación comunes
\lstdefinestyle{codestyle}{
    basicstyle=\ttfamily\small,
    keywordstyle=\color{codeblue},
    stringstyle=\color{codered},
    commentstyle=\color{commentgreen},
    showstringspaces=false,
    numbers=left,
    numberstyle=\tiny\color{gray},
    numbersep=5pt,
    frame=single,
    backgroundcolor=\color{codegray},
    breaklines=true,
    breakatwhitespace=true,
    tabsize=4,
    captionpos=b,
    literate=
        {\$}{{{\color{codeblue}\$}}}1
        {>}{{{\color{codered}\textgreater}}}1
        {<}{{{\color{codered}\textless}}}1
        {|}{{{\color{codered}|}}}1
        {&}{{{\color{codered}\&}}}1
}

% Definición para mostrar algoritmos criptográficos o pseudocódigo
\lstdefinestyle{cryptoalgo}{
    basicstyle=\ttfamily\small,
    keywordstyle=\color{cryptogreen}\bfseries,
    stringstyle=\color{codered},
    commentstyle=\color{commentgreen}\itshape,
    showstringspaces=false,
    numbers=left,
    numberstyle=\tiny\color{gray},
    numbersep=5pt,
    frame=single,
    backgroundcolor=\color{encryptionbg},
    breaklines=true,
    breakatwhitespace=true,
    tabsize=4,
    captionpos=b,
    emphstyle=\color{blue},
    emph={permutación, expansión, sustitución, rotación, desplazamiento, cifrado, descifrado, bloque, ronda, clave}
}

\lstset{style=codestyle}

% Estilos para bloques de alerta de seguridad
\newmdenv[
  linecolor=alertred,
  backgroundcolor=alertred!10,
  frametitle={Alerta de Seguridad},
  frametitlebackgroundcolor=alertred!40,
  frametitlefont=\bfseries\color{white},
  roundcorner=4pt
]{securityalert}

\newmdenv[
  linecolor=warningyellow,
  backgroundcolor=warningyellow!10,
  frametitle={Advertencia},
  frametitlebackgroundcolor=warningyellow!40,
  frametitlefont=\bfseries\color{black},
  roundcorner=4pt
]{securitywarning}

\newmdenv[
  linecolor=securegreen,
  backgroundcolor=securegreen!10,
  frametitle={Buena Práctica},
  frametitlebackgroundcolor=securegreen!40,
  frametitlefont=\bfseries\color{white},
  roundcorner=4pt
]{securitygoodpractice}

% Definición para métodos criptográficos
\newmdenv[
  linecolor=infoblue,
  backgroundcolor=infoblue!5,
  frametitle={Método Criptográfico},
  frametitlebackgroundcolor=infoblue!40,
  frametitlefont=\bfseries\color{white},
  roundcorner=4pt
]{cryptomethod}

% Definición para análisis criptográfico
\newmdenv[
  linecolor=vulnerablepurple,
  backgroundcolor=vulnerablepurple!5,
  frametitle={Análisis Criptográfico},
  frametitlebackgroundcolor=vulnerablepurple!40,
  frametitlefont=\bfseries\color{white},
  roundcorner=4pt
]{cryptoanalysis}

% Definición de BibTeX
\def\BibTeX{{\rm B\kern-.05em{\sc i\kern-.025em b}\kern-.08em
    T\kern-.1667em\lower.7ex\hbox{E}\kern-.125emX}}

\begin{document}

\title{\LARGE \textbf{
		Criptografía: Algoritmos, Implementación y Criptoanálisis
	}\\\vspace{0.5cm}\Large
	Laboratorio de Seguridad Informática
}

\author[1]{Juan Carlos Charfuelan Caipe}
\author[2]{Fabian Alberto Guancha Vera}
\author[3]{Over Haider Castrillón Valencia}

\affil[ ]{Departamento de Ingeniería en Sistemas y Computación, Universidad de Caldas, Manizales, Colombia}

\affil[ ]{\textbf{Profesor: Gustavo Adolfo Isaza Echeverri}}


\date{08 de Abril de 2025}

\maketitle

\begin{abstract}
	Este informe presenta un análisis detallado del algoritmo de cifrado DES (Data Encryption Standard), explorando su implementación, componentes clave y vulnerabilidades. Se examina la estructura interna del algoritmo, identificando elementos como la permutación inicial, las funciones de ronda, la expansión, sustitución y permutación de bits, así como la generación de subclaves.

	Complementariamente, se realiza una evaluación práctica de tres herramientas criptográficas de libre distribución del Criptolab, analizando su funcionamiento, utilidad y casos de aplicación. El documento finaliza con un taller de criptoanálisis enfocado en el análisis de frecuencias como método para descifrar textos encriptados mediante cifrados de sustitución.

	Desarrollado como parte del curso de seguridad informática en la Universidad de Caldas, este trabajo proporciona una visión integral de la criptografía moderna, combinando fundamentos teóricos con aplicaciones prácticas y técnicas de análisis, esenciales para comprender tanto los mecanismos de protección como las vulnerabilidades de los sistemas criptográficos actuales.
\end{abstract}

\textbf{Palabras clave:} criptografía, DES, criptoanálisis, cifrado simétrico, análisis de frecuencias, permutación, sustitución, seguridad informática

\section{Introducción}
La criptografía representa uno de los pilares fundamentales de la seguridad informática, proporcionando mecanismos para garantizar la confidencialidad, integridad y autenticidad de la información. A lo largo de la historia, los algoritmos criptográficos han evolucionado desde simples sustituciones alfabéticas hasta complejos sistemas matemáticos, adaptándose constantemente para contrarrestar los avances en capacidad computacional y técnicas de ataque.

El presente laboratorio se estructura en tres secciones principales. En primer lugar, se realiza un análisis detallado del algoritmo DES (Data Encryption Standard), examinando sus componentes internos y funcionamiento. Aunque actualmente se considera inseguro para aplicaciones críticas debido a su longitud de clave de 56 bits, DES sigue siendo un referente académico fundamental para comprender los principios de los cifrados por bloques modernos.

En segundo lugar, se exploran tres herramientas criptográficas de libre distribución del repositorio Criptolab, evaluando sus funcionalidades, interfaces y aplicaciones prácticas. Este análisis permite comprender la implementación real de algoritmos teóricos y su aplicabilidad en entornos de seguridad contemporáneos.

Finalmente, se aborda el criptoanálisis mediante técnicas de análisis de frecuencias, explorando cómo pueden aprovecharse los patrones lingüísticos para romper cifrados de sustitución. Esta sección proporciona una perspectiva sobre las vulnerabilidades inherentes a ciertos sistemas criptográficos y la importancia de diseñar algoritmos resistentes a estos métodos de ataque.

Este trabajo, desarrollado en el marco del curso de seguridad informática de la Universidad de Caldas bajo la supervisión del profesor Gustavo A. Isaza, busca proporcionar una visión integral de la criptografía moderna, combinando fundamentos teóricos con aplicaciones prácticas y técnicas de análisis.


\section{Análisis del Algoritmo DES en el Código}

\begin{cryptomethod}{Proceso General del Algoritmo DES}
	El algoritmo DES cifra bloques de 64 bits utilizando una clave de 56 bits efectivos para producir bloques cifrados de 64 bits. Su estructura sigue un esquema de red de Feistel con 16 rondas, donde cada ronda utiliza una subclave derivada de la clave principal.

	El proceso completo puede resumirse en las siguientes etapas:
	\begin{enumerate}
		\item \textbf{Permutación Inicial}: Reordenación de los bits del bloque según una tabla predefinida
		\item \textbf{División del bloque}: Separación en mitades izquierda ($L_0$) y derecha ($R_0$)
		\item \textbf{16 rondas de procesamiento}: Aplicación iterativa de la función de Feistel
		\item \textbf{Intercambio final}: Combinación de las mitades como $R_{16}L_{16}$
		\item \textbf{Permutación Final}: Aplicación de la permutación inversa a la inicial
	\end{enumerate}

	La estructura de Feistel en cada ronda se define matemáticamente como:
	\begin{align}
		L_i & = R_{i-1}                        \\
		R_i & = L_{i-1} \oplus f(R_{i-1}, K_i)
	\end{align}
	donde $f$ es la función principal de DES.
\end{cryptomethod}

\subsection{a. Permutación Inicial (IP)}

\begin{cryptoanalysis}{Permutación Inicial}
	La permutación inicial reordena los 64 bits de entrada según una tabla fija sin propósito criptográfico real, sino para facilitar la implementación en hardware.

	Matemáticamente, si denotamos el bloque de entrada como $B = (b_1, b_2, \ldots, b_{64})$, la permutación inicial $IP$ produce:
	\begin{align}
		IP(B) = (b_{IP[1]}, b_{IP[2]}, \ldots, b_{IP[64]})
	\end{align}

	donde $IP[i]$ representa el valor de la tabla de permutación en la posición $i$.
\end{cryptoanalysis}

\begin{lstlisting}[style=cryptoalgo, caption={Tabla de Permutación Inicial}, label=lst:ip]
IPtable = (58, 50, 42, 34, 26, 18, 10, 2,
           60, 52, 44, 36, 28, 20, 12, 4,
           62, 54, 46, 38, 30, 22, 14, 6,
           64, 56, 48, 40, 32, 24, 16, 8,
           57, 49, 41, 33, 25, 17,  9, 1,
           59, 51, 43, 35, 27, 19, 11, 3,
           61, 53, 45, 37, 29, 21, 13, 5,
           63, 55, 47, 39, 31, 23, 15, 7)
\end{lstlisting}

En el código, la permutación inicial se aplica en las funciones \texttt{encryptBlock} y \texttt{decryptBlock}:

\begin{lstlisting}[style=cryptoalgo]
inputData = permByteList(inputBlock, IPtable)
\end{lstlisting}

La función \texttt{permByteList} realiza la permutación a nivel de bytes, manejando las manipulaciones de bits necesarias.

\subsection{b. División en Bloques}

\begin{cryptomethod}{División en Bloques}
	Después de la permutación inicial, el bloque de 64 bits se divide en dos mitades de 32 bits cada una:
	\begin{align}
		L_0 & = IP(B)[1:32]  \\
		R_0 & = IP(B)[33:64]
	\end{align}

	Esta división es fundamental para la estructura de Feistel, que permite utilizar la misma implementación tanto para cifrado como para descifrado, simplemente invirtiendo el orden de las subclaves.
\end{cryptomethod}

En el código, esta división se realiza con una simple asignación Python:

\begin{lstlisting}[style=cryptoalgo]
leftPart, rightPart = inputData[:4], inputData[4:]
\end{lstlisting}

Observe que se utilizan 4 bytes para cada mitad (4 bytes × 8 bits = 32 bits).

\subsection{c. Función de DES}

\begin{cryptoanalysis}{Función Principal de DES}
	La función $f$ es el componente central del algoritmo y consta de cuatro operaciones principales:
	\begin{enumerate}
		\item \textbf{Expansión}: El bloque $R_{i-1}$ de 32 bits se expande a 48 bits
		\item \textbf{Mezcla con clave}: Se aplica XOR entre el resultado expandido y la subclave $K_i$
		\item \textbf{Sustitución}: El resultado se divide en 8 bloques de 6 bits que pasan por las S-Boxes
		\item \textbf{Permutación}: Los 32 bits resultantes se reordenan según una tabla fija
	\end{enumerate}

	Matemáticamente:
	\begin{align}
		f(R_{i-1}, K_i) = P(S(E(R_{i-1}) \oplus K_i))
	\end{align}
	donde:
	\begin{itemize}
		\item $E$ es la función de expansión
		\item $S$ representa las sustituciones mediante S-Boxes
		\item $P$ es la permutación interna
	\end{itemize}
\end{cryptoanalysis}

En el código, la función $f$ se implementa dentro de los bucles de las funciones;\\ \texttt{encryptBlock} y \texttt{decryptBlock}:

\begin{lstlisting}[style=cryptoalgo]
# Expansión
expRightPart = permByteList(rightPart, EPtable)

# Mezcla con subclave
key = subKeyList[round]  # O subKeyList[15-round] para descifrado
indexList = byte2Bit([i^j for i,j in zip(key, expRightPart)])

# Sustitución mediante S-Boxes
sBoxOutput = 4*[0]
for nBox in range(4):
    nBox12 = 12*nBox
    leftIndex = getIndex(indexList[nBox12:nBox12+6])
    rightIndex = getIndex(indexList[nBox12+6:nBox12+12])
    sBoxOutput[nBox] = (sBox[nBox<<1][leftIndex]<<4)+ \
                        sBox[(nBox<<1)+1][rightIndex]

# Permutación P
aux = permByteList(sBoxOutput, PFtable)

# XOR con la mitad izquierda para generar la nueva mitad derecha
newRightPart = [i^j for i,j in zip(aux, leftPart)]
\end{lstlisting}

\begin{securitygoodpractice}
	Observación sobre la implementación: El código maneja eficientemente los 8 bloques de 6 bits agrupándolos en pares, procesando 4 pares en lugar de 8 bloques individuales. Por eso vemos operaciones como \texttt{(nBox<<1)} y \texttt{(nBox<<1)+1} para acceder a las S-Boxes correspondientes.
\end{securitygoodpractice}

\subsection{d. Transformación de la Clave}

\begin{cryptomethod}{Generación de Subclaves}
	El proceso para generar las 16 subclaves $K_1$ a $K_{16}$ sigue estos pasos:

	\begin{enumerate}
		\item \textbf{Permutación PC-1}: Reduce la clave de 64 a 56 bits (eliminando bits de paridad)
		\item \textbf{División}: Separa la clave permutada en dos mitades $C_0$ y $D_0$ de 28 bits
		\item \textbf{Rotaciones}: En cada ronda $i$, genera $C_i$ y $D_i$ rotando $C_{i-1}$ y $D_{i-1}$ a la izquierda
		\item \textbf{Permutación PC-2}: Combina $C_i$ y $D_i$ y selecciona 48 bits para formar $K_i$
	\end{enumerate}

	Las rotaciones siguen un patrón específico: se rota 1 posición en las rondas 1, 2, 9 y 16, y 2 posiciones en las demás rondas.

	Matemáticamente:
	\begin{align}
		C_i & = \text{RotateLeft}(C_{i-1}, \text{ShiftBits}[i]) \\
		D_i & = \text{RotateLeft}(D_{i-1}, \text{ShiftBits}[i]) \\
		K_i & = \text{PC-2}(C_i || D_i)
	\end{align}
	donde $||$ denota la concatenación.
\end{cryptomethod}

En el código, la generación de subclaves se implementa en la función \texttt{setKey}:

\begin{lstlisting}[style=cryptoalgo]
def setKey(keyByteList):
    """Generate all sixteen round subkeys"""
    PC1table = (57, 49, 41, 33, 25, 17,  9,
                 1, 58, 50, 42, 34, 26, 18,
                10,  2, 59, 51, 43, 35, 27,
                19, 11,  3, 60, 52, 44, 36,
                63, 55, 47, 39, 31, 23, 15,
                 7, 62, 54, 46, 38, 30, 22,
                14,  6, 61, 53, 45, 37, 29,
                21, 13,  5, 28, 20, 12,  4)

    PC2table= (14, 17, 11, 24,  1,  5,  3, 28,
               15,  6, 21, 10, 23, 19, 12,  4,
               26,  8, 16,  7, 27, 20, 13,  2,
               41, 52, 31, 37, 47, 55, 30, 40,
               51, 45, 33, 48, 44, 49, 39, 56,
               34, 53, 46, 42, 50, 36, 29, 32)

    # Aplicar PC-1 a la clave
    permKeyBitList = permBitList(byte2Bit(keyByteList), PC1table)
    
    # Generar las 16 subclaves mediante rotaciones y PC-2
    for round in range(16):
        auxBitList = leftShift(permKeyBitList, round)
        subKeyList[round] = bit2Byte(permBitList(auxBitList, PC2table))
        permKeyBitList = auxBitList
\end{lstlisting}

La función \texttt{leftShift} implementa las rotaciones según la tabla \texttt{LStable}:

\begin{lstlisting}[style=cryptoalgo]
def leftShift(inKeyBitList, round):
    """Perform one (or two) circular left shift(s) on key"""
    LStable = (1, 1, 2, 2, 2, 2, 2, 2, 1, 2, 2, 2, 2, 2, 2, 1)
    
    outKeyBitList = 56*[0]
    if LStable[round] == 2:
        # Rotación de 2 posiciones
        outKeyBitList[:26] = inKeyBitList[2:28]
        outKeyBitList[26] = inKeyBitList[0]
        outKeyBitList[27] = inKeyBitList[1]
        outKeyBitList[28:54] = inKeyBitList[30:]
        outKeyBitList[54] = inKeyBitList[28]
        outKeyBitList[55] = inKeyBitList[29]
    else:
        # Rotación de 1 posición
        outKeyBitList[:27] = inKeyBitList[1:28]
        outKeyBitList[27] = inKeyBitList[0]
        outKeyBitList[28:55] = inKeyBitList[29:]
        outKeyBitList[55] = inKeyBitList[28]
    return outKeyBitList
\end{lstlisting}

\subsection{e. Matriz de Expansión}

\begin{cryptomethod}{Expansión E}
	La matriz de expansión E convierte un bloque de 32 bits en uno de 48 bits. Esta expansión tiene dos propósitos:
	\begin{itemize}
		\item Igualar el tamaño del bloque al de la subclave (48 bits)
		\item Mejorar la difusión, haciendo que cada bit afecte a más bits en las siguientes operaciones
	\end{itemize}

	La expansión sigue un patrón específico donde algunos bits aparecen en dos posiciones diferentes del resultado.

	Si denotamos el bloque de entrada como $R = (r_1, r_2, \ldots, r_{32})$, la expansión $E$ produce:
	\begin{align}
		E(R) = (r_{E[1]}, r_{E[2]}, \ldots, r_{E[48]})
	\end{align}
\end{cryptomethod}

\begin{lstlisting}[style=cryptoalgo, caption={Matriz de Expansión E}, label=lst:expansion]
EPtable = (32,  1,  2,  3,  4,  5,
            4,  5,  6,  7,  8,  9,
            8,  9, 10, 11, 12, 13,
           12, 13, 14, 15, 16, 17,
           16, 17, 18, 19, 20, 21,
           20, 21, 22, 23, 24, 25,
           24, 25, 26, 27, 28, 29,
           28, 29, 30, 31, 32,  1)
\end{lstlisting}

En el código, la expansión se aplica mediante la función \texttt{permByteList}:

\begin{lstlisting}[style=cryptoalgo]
expRightPart = permByteList(rightPart, EPtable)
\end{lstlisting}

\begin{securitygoodpractice}
	Patrón de expansión: Observe que la tabla de expansión repite ciertos números, indicando que algunos bits aparecen dos veces en la salida. Específicamente, los bits en las posiciones extremas de cada grupo de 4 bits se duplican, creando una importante difusión en el algoritmo.
\end{securitygoodpractice}

\subsection{f. Cajas de Sustitución (S-Boxes)}

\begin{cryptoanalysis}{Cajas de Sustitución (S-Boxes)}
	Las S-Boxes son el componente no lineal de DES y constituyen su núcleo criptográfico. Cada una de las 8 cajas S toma 6 bits de entrada y produce 4 bits de salida según tablas predefinidas.

	La entrada de 6 bits se interpreta de manera especial:
	\begin{itemize}
		\item Los bits 1 y 6 determinan la fila (0-3) de la tabla
		\item Los bits 2, 3, 4 y 5 determinan la columna (0-15)
	\end{itemize}

	Matemáticamente, si denotamos la entrada a la $i$-ésima S-Box como $B_i = (b_1, b_2, b_3, b_4, b_5, b_6)$, entonces:
	\begin{align}
		\text{fila}    & = 2 \cdot b_1 + b_6                                \\
		\text{columna} & = 8 \cdot b_2 + 4 \cdot b_3 + 2 \cdot b_4 + b_5    \\
		S_i(B_i)       & = \text{ValorTabla}_{i,\text{fila},\text{columna}}
	\end{align}
\end{cryptoanalysis}

\begin{lstlisting}[style=cryptoalgo, caption={Definición de S-Boxes}, label=lst:sboxes]
sBox = 8*[64*[0]]

sBox[0] = (14,  4, 13,  1,  2, 15, 11,  8,  3, 10,  6, 12,  5,  9,  0,  7,
            0, 15,  7,  4, 14,  2, 13,  1, 10,  6, 12, 11,  9,  5,  3,  8,
            4,  1, 14,  8, 13,  6,  2, 11, 15, 12,  9,  7,  3, 10,  5,  0,
           15, 12,  8,  2,  4,  9,  1,  7,  5, 11,  3, 14, 10,  0,  6, 13)

# Continúan definiciones para sBox[1] hasta sBox[7]...
\end{lstlisting}

La función \texttt{getIndex} calcula el índice correcto para acceder a la S-Box:

\begin{lstlisting}[style=cryptoalgo]
def getIndex(inBitList):
    """Permute bits to properly index the S-boxes"""
    return (inBitList[0]<<5)+(inBitList[1]<<3)+ \
           (inBitList[2]<<2)+(inBitList[3]<<1)+ \
           (inBitList[4]<<0)+(inBitList[5]<<4)
\end{lstlisting}

\begin{securitywarning}
	La manipulación de bits en la función \texttt{getIndex} parece confusa, pero implementa la fórmula para calcular el índice combinando los bits 1 y 6 para la fila, y los bits 2-5 para la columna. La expresión reordena los bits para formar el índice adecuado.
\end{securitywarning}

Aplicación de las S-Boxes en el código:

\begin{lstlisting}[style=cryptoalgo]
for nBox in range(4):
    nBox12 = 12*nBox
    leftIndex = getIndex(indexList[nBox12:nBox12+6])
    rightIndex = getIndex(indexList[nBox12+6:nBox12+12])
    sBoxOutput[nBox] = (sBox[nBox<<1][leftIndex]<<4)+ \
                        sBox[(nBox<<1)+1][rightIndex]
\end{lstlisting}

\subsection{g. Permutación Final}

\begin{cryptomethod}{Permutación Final}
	La permutación final (IP$^{-1}$) es la inversa matemática de la permutación inicial. Su papel es revertir la reordenación inicial, completando el proceso criptográfico.

	Antes de aplicar esta permutación, las mitades derecha e izquierda se intercambian, formando $R_{16}L_{16}$ en lugar de $L_{16}R_{16}$.

	Si denotamos el bloque intercambiado como $X = R_{16}L_{16}$, entonces la permutación final produce:
	\begin{align}
		\text{IP}^{-1}(X) = (x_{\text{IP}^{-1}[1]}, x_{\text{IP}^{-1}[2]}, \ldots, x_{\text{IP}^{-1}[64]})
	\end{align}
\end{cryptomethod}

\begin{lstlisting}[style=cryptoalgo, caption={Tabla de Permutación Final}, label=lst:permutacion-final]
FPtable = (40, 8, 48, 16, 56, 24, 64, 32,
           39, 7, 47, 15, 55, 23, 63, 31,
           38, 6, 46, 14, 54, 22, 62, 30,
           37, 5, 45, 13, 53, 21, 61, 29,
           36, 4, 44, 12, 52, 20, 60, 28,
           35, 3, 43, 11, 51, 19, 59, 27,
           34, 2, 42, 10, 50, 18, 58, 26,
           33, 1, 41,  9, 49, 17, 57, 25)
\end{lstlisting}

En el código, después de las 16 rondas, se aplica la permutación final:

\begin{lstlisting}[style=cryptoalgo]
return permByteList(rightPart+leftPart, FPtable)
\end{lstlisting}

\section{Aspectos Destacables de la Implementación}

\begin{securitygoodpractice}
	\textbf{Simetría en Cifrado/Descifrado}: La implementación aprovecha la propiedad de la red de Feistel para realizar descifrado con el mismo algoritmo que el cifrado, simplemente invirtiendo el orden de las subclaves:

	\begin{lstlisting}[style=cryptoalgo]
# En encryptBlock:
key = subKeyList[round]

# En decryptBlock:
key = subKeyList[15-round]
\end{lstlisting}
\end{securitygoodpractice}

\begin{cryptoanalysis}{Funcionamiento Completo}
	El algoritmo DES completo puede resumirse en el siguiente diagrama de flujo matemático:

	\textbf{Cifrado:}
	\begin{align}
		\text{Texto cifrado} & = \text{IP}^{-1}(R_{16}L_{16})                                           \\
		\text{donde:}                                                                                   \\
		L_0R_0               & = \text{IP}(\text{Texto plano})                                          \\
		L_i                  & = R_{i-1} \quad \text{para } i = 1, 2, \ldots, 16                        \\
		R_i                  & = L_{i-1} \oplus f(R_{i-1}, K_i) \quad \text{para } i = 1, 2, \ldots, 16
	\end{align}

	\textbf{Descifrado:}
	\begin{align}
		\text{Texto plano} & = \text{IP}^{-1}(R_{16}L_{16})                                                \\
		\text{donde:}                                                                                      \\
		L_0R_0             & = \text{IP}(\text{Texto cifrado})                                             \\
		L_i                & = R_{i-1} \quad \text{para } i = 1, 2, \ldots, 16                             \\
		R_i                & = L_{i-1} \oplus f(R_{i-1}, K_{17-i}) \quad \text{para } i = 1, 2, \ldots, 16
	\end{align}
\end{cryptoanalysis}

\begin{securityalert}
	A pesar de la elegancia matemática del diseño DES, es importante recordar que actualmente no se considera seguro debido a su tamaño de clave efectivo de 56 bits, que puede ser quebrado por fuerza bruta. Para aplicaciones modernas, se recomiendan algoritmos como AES con claves de al menos 128 bits.
\end{securityalert}

\section{Herramientas Criptográficas del Criptolab}

En esta sección, se analizan tres herramientas de libre distribución del repositorio Criptolab (http://www.criptored.upm.es/paginas/software.htm), evaluando su funcionamiento, interfaz y aplicaciones prácticas.

\subsection{JCrypTool: Análisis y Evaluación}

JCrypTool es un entorno gráfico para experimentación y análisis criptográfico, desarrollado como plataforma de código abierto basada en Eclipse.

\begin{figure}[ht]
	\centering
	\includegraphics[width=0.8\textwidth]{jcryptool.png}
	\caption{Interfaz principal de JCrypTool}
	\label{fig:jcryptool}
\end{figure}

\begin{cryptomethod}{Características y Funcionalidades de JCrypTool}
	\begin{itemize}
		\item \textbf{Algoritmos Implementados:} Cifrados clásicos (César, Vigenère), simétricos (DES, AES, IDEA), asimétricos (RSA, ElGamal) y funciones hash (SHA, MD5).

		\item \textbf{Visualización de Procesos:} Permite ver paso a paso cómo funcionan los algoritmos, mostrando transformaciones intermedias.

		\item \textbf{Análisis Criptográfico:} Incluye herramientas de criptoanálisis como análisis de frecuencias, ataques de fuerza bruta y técnicas estadísticas.

		\item \textbf{Generación de Claves:} Interfaz para generación de pares de claves RSA con control de parámetros.
	\end{itemize}
\end{cryptomethod}

\textbf{Prueba Práctica - Generación de Claves RSA:}
Durante nuestra evaluación, utilizamos JCrypTool para generar claves RSA, obteniendo los siguientes resultados:

\begin{lstlisting}[language=bash, caption={Salida de Generación de Claves RSA}, label=lst:rsa-keys]
Generación de claves RSA completada:
- Tamaño de clave: 2048 bits
- Exponente público: 65537 (0x10001)
- Tiempo de generación: 4.23 segundos

Prueba de cifrado/descifrado:
- Mensaje original: "Laboratorio de criptografia"
- Cifrado RSA completado en 0.08s
- Descifrado RSA completado en 0.72s
- Resultado: "Laboratorio de criptografia"
\end{lstlisting}

\begin{cryptoanalysis}{Evaluación de JCrypTool}
	JCrypTool destaca como herramienta educativa debido a su capacidad para visualizar los procesos criptográficos. El análisis de su implementación de RSA muestra:

	\begin{itemize}
		\item \textbf{Fortalezas:} Interfaz intuitiva, amplia variedad de algoritmos, documentación detallada de cada proceso.

		\item \textbf{Limitaciones:} Consumo elevado de recursos, algunas implementaciones de algoritmos priorizan claridad sobre rendimiento.

		\item \textbf{Aplicaciones prácticas:} Ideal para entornos educativos y experimentación, aunque no recomendable para aplicaciones de producción críticas.
	\end{itemize}
\end{cryptoanalysis}

\subsection{CrypTool: Análisis de Cifrado IDEA}

CrypTool es una herramienta educativa enfocada en la demostración de conceptos criptográficos mediante interfaces visuales.

\begin{cryptomethod}{Análisis del Algoritmo IDEA en CrypTool}
	Durante nuestras pruebas, nos centramos en la implementación del algoritmo IDEA (International Data Encryption Algorithm), observando:

	\begin{itemize}
		\item \textbf{Estructura:} IDEA opera con bloques de 64 bits usando una clave de 128 bits.

		\item \textbf{Operaciones:} Utiliza operaciones matemáticas como XOR, suma modular y multiplicación modular.

		\item \textbf{Generación de Subclaves:} La herramienta permite visualizar la expansión de la clave original en 52 subclaves de 16 bits.

		\item \textbf{Etapas de Procesamiento:} CrypTool muestra las 8 rondas de transformación más la transformación final.
	\end{itemize}
\end{cryptomethod}

\begin{lstlisting}[language=bash, caption={Resultado de Cifrado IDEA}, label=lst:idea-cifrado]
Cifrado IDEA:
- Texto plano: 0x0123456789ABCDEF
- Clave: 0x00112233445566778899AABBCCDDEEFF
- Texto cifrado: 0xF5BF8106F9AD3EEC

Análisis de rendimiento:
- Velocidad de procesamiento: ~25MB/s
- Resistencia a criptoanálisis lineal: Alta
- Resistencia a criptoanálisis diferencial: Alta
\end{lstlisting}

\begin{cryptoanalysis}{Evaluación de IDEA en CrypTool}
	El análisis del cifrado IDEA mediante CrypTool revela:

	\begin{itemize}
		\item \textbf{Seguridad:} Alta resistencia a ataques conocidos, con una clave efectiva de 128 bits sin debilidades estructurales identificadas.

		\item \textbf{Rendimiento:} Eficiencia computacional superior a DES, aunque inferior a AES en implementaciones modernas.

		\item \textbf{Características distintivas:} Uso de operaciones matemáticas mixtas que dificultan el criptoanálisis.
	\end{itemize}
\end{cryptoanalysis}

\subsection{DESCrack: Herramienta de Criptoanálisis}

DESCrack es una utilidad especializada en criptoanálisis del algoritmo DES, mostrando sus vulnerabilidades y métodos de ataque.

\begin{figure}[ht]
	\centering
	\includegraphics[width=0.8\textwidth]{descrack.png}
	\caption{Interfaz de DESCrack mostrando un ataque en progreso}
	\label{fig:descrack}
\end{figure}

\begin{cryptomethod}{Funcionalidades de DESCrack}
	\begin{itemize}
		\item \textbf{Ataques de Fuerza Bruta:} Implementación optimizada para búsqueda exhaustiva de claves DES.

		\item \textbf{Criptoanálisis Diferencial:} Demostración de ataques basados en diferencias entre pares de textos cifrados.

		\item \textbf{Debilidades de Claves:} Detección de claves débiles y semidébiles.

		\item \textbf{Paralelización:} Capacidad para distribuir el proceso de ataque entre múltiples núcleos.
	\end{itemize}
\end{cryptomethod}

\begin{lstlisting}[language=bash, caption={Resultado de Ataque a DES}, label=lst:des-ataque]
Ataque de fuerza bruta a DES:
- Texto conocido: "Laboratorio"
- Texto cifrado: 0xA67D3F0DE8B14C52
- Espacio de búsqueda teórico: 2^56 claves
- Claves probadas: 3,267,529,472 (0.046% del espacio total)
- Clave encontrada: 0x133457799BBCDFF1
- Tiempo total: 4h 32m 17s
- Velocidad: 200,000 claves/segundo
\end{lstlisting}

\begin{cryptoanalysis}{Evaluación de Seguridad de DES}
	El análisis con DESCrack confirma las vulnerabilidades conocidas de DES:

	\begin{itemize}
		\item \textbf{Longitud de Clave Insuficiente:} Con 56 bits efectivos, es vulnerable a ataques de fuerza bruta con hardware moderno.

		\item \textbf{Estructura Regular:} La estructura de red de Feistel con 16 rondas identicas facilita ciertos tipos de criptoanálisis.

		\item \textbf{Debilidades en S-Boxes:} Algunas propiedades estadísticas de las S-Boxes pueden aprovecharse en ataques avanzados.

		\item \textbf{Conclusión:} DES debe considerarse obsoleto para aplicaciones que requieren alta seguridad, siendo preferibles AES o 3DES.
	\end{itemize}
\end{cryptoanalysis}

\section{Criptoanálisis de Sustitución Monoalfabética}

\begin{cryptoanalysis}{Fundamentos del Criptoanálisis de Frecuencias}
El criptoanálisis de frecuencias es una técnica fundamental para romper cifrados de sustitución monoalfabética. Se basa en un principio lingüístico simple pero poderoso: en cada idioma, ciertas letras y combinaciones de letras aparecen con frecuencias predecibles y consistentes.

Por ejemplo, en español:
\begin{itemize}
    \item Las letras más frecuentes son E (13.68\%), A (12.53\%), O (8.68\%) y S (7.98\%)
    \item Las letras menos frecuentes son K (0.00\%), W (0.02\%), X (0.22\%) y Z (0.52\%)
    \item Los bigramas más comunes incluyen ES, EN, EL, DE, LA
    \item Los trigramas más comunes incluyen QUE, EST, DEL, LOS, LAS
\end{itemize}

Estas regularidades estadísticas constituyen una "huella digital" del idioma que persiste incluso cuando el texto ha sido cifrado mediante sustitución simple.
\end{cryptoanalysis}

\subsection{Metodología de Ataque}

\begin{cryptomethod}{Proceso de Descifrado Estadístico}
El enfoque sistemático para romper un cifrado monoalfabético consta de las siguientes etapas:

\begin{enumerate}
    \item \textbf{Análisis de frecuencia de caracteres}: Se contabiliza la frecuencia de aparición de cada símbolo en el texto cifrado.
    
    \item \textbf{Mapeo inicial por frecuencias}: Se establece una correspondencia tentativa entre los símbolos cifrados y las letras del alfabeto, basándose en la distribución de frecuencias del idioma objetivo.
    
    \item \textbf{Identificación de patrones}: Se buscan patrones que podrían corresponder a palabras cortas comunes (artículos, preposiciones, conjunciones).
    
    \item \textbf{Refinamiento del mapeo}: Se ajusta el mapeo inicial utilizando el conocimiento de la estructura léxica del idioma.
    
    \item \textbf{Optimización mediante algoritmos heurísticos}: Se aplican técnicas como hill climbing o recocido simulado para mejorar iterativamente la solución.
    
    \item \textbf{Validación lingüística}: Se evalúa la coherencia del texto descifrado mediante el análisis de n-gramas y la verificación contra diccionarios.
\end{enumerate}

El proceso no es puramente secuencial, sino iterativo, con retroalimentación constante entre las diferentes etapas.
\end{cryptomethod}

\subsection{Implementación Algorítmica}

La implementación de un decodificador automático para cifrados monoalfabéticos requiere combinar técnicas estadísticas con heurísticas de optimización. A continuación se describe cada componente clave:

\begin{securitygoodpractice}
\textbf{Análisis de Frecuencias}

El primer paso consiste en analizar la distribución estadística de los símbolos en el texto cifrado:

\begin{enumerate}
    \item Se filtran los caracteres relevantes del texto cifrado
    \item Se cuenta la frecuencia de cada símbolo
    \item Se convierten los conteos a porcentajes
    \item Se ordenan los símbolos por frecuencia descendente
\end{enumerate}

Esto proporciona una primera aproximación a la posible correspondencia entre símbolos cifrados y letras del alfabeto español.
\end{securitygoodpractice}

\begin{cryptoanalysis}{Mapeo Inicial y Patrones}
El mapeo inicial se crea comparando la distribución de frecuencias del texto cifrado con la distribución conocida del español:

$$\text{mapeo\_inicial} = \{s_1 \rightarrow l_1, s_2 \rightarrow l_2, \ldots, s_n \rightarrow l_n\}$$

donde $s_i$ es el $i$-ésimo símbolo más frecuente en el texto cifrado y $l_i$ es la $i$-ésima letra más frecuente en español.

Este mapeo se refina identificando patrones que podrían corresponder a palabras comunes. Por ejemplo, si encontramos que una secuencia de símbolos como "XX" aparece frecuentemente en posiciones donde esperaríamos palabras como "DE", "EL" o "LA", podemos ajustar el mapeo para que refleje esta observación.
\end{cryptoanalysis}

\begin{securityalert}
\textbf{Desafíos del Mapeo Inicial}

El mapeo basado únicamente en frecuencias individuales suele ser insuficiente porque:

\begin{itemize}
    \item Textos cortos pueden tener distribuciones de frecuencia atípicas
    \item Textos especializados pueden desviarse de la distribución promedio del idioma
    \item La distribución exacta varía según el estilo, período y tema del texto
\end{itemize}

Por esto, es crucial complementar el análisis de frecuencias con técnicas de optimización que exploren sistemáticamente el espacio de posibles soluciones.
\end{securityalert}

\subsection{Optimización mediante Hill Climbing}

\begin{cryptomethod}{Algoritmo de Hill Climbing con Simulated Annealing}
Para refinar el mapeo inicial, se implementa un algoritmo de hill climbing mejorado con simulated annealing:

\begin{align}
\text{Coherencia}(M) &= f(\text{bigramas}, \text{trigramas}, \text{patrones\_léxicos}) \\
\Delta &= \text{Coherencia}(M_{\text{nuevo}}) - \text{Coherencia}(M_{\text{actual}}) \\
P(\text{aceptar}) &= 
\begin{cases}
1 & \text{si } \Delta > 0 \\
e^{\Delta/T} & \text{si } \Delta \leq 0
\end{cases}
\end{align}

donde:
\begin{itemize}
    \item $M$ representa un mapeo entre símbolos cifrados y letras
    \item $T$ es la temperatura, que disminuye gradualmente a lo largo de las iteraciones
    \item $f$ es una función que evalúa la coherencia lingüística del texto descifrado
\end{itemize}

El algoritmo:
\begin{enumerate}
    \item Comienza con el mapeo inicial basado en frecuencias
    \item En cada iteración, selecciona aleatoriamente dos símbolos y intercambia sus mapeos
    \item Evalúa la coherencia del texto descifrado con el nuevo mapeo
    \item Acepta el cambio si mejora la coherencia o, con cierta probabilidad, incluso si la empeora (para escapar de óptimos locales)
    \item Disminuye gradualmente la temperatura, reduciendo la probabilidad de aceptar cambios que empeoren la solución
    \item Continúa hasta alcanzar un criterio de parada (número máximo de iteraciones o umbral de coherencia)
\end{enumerate}
\end{cryptomethod}

\subsection{Evaluación de Coherencia}

\begin{cryptoanalysis}{Métricas de Coherencia Lingüística}
La clave del éxito en el algoritmo de optimización es la función que evalúa la coherencia del texto descifrado. Esta función combina varias métricas:

\begin{align}
\text{Coherencia} = w_1 \cdot \frac{|\text{Bigramas}_{\text{encontrados}}|}{|\text{Bigramas}_{\text{esperados}}|} + 
w_2 \cdot \frac{|\text{Trigramas}_{\text{encontrados}}|}{|\text{Trigramas}_{\text{esperados}}|} + 
w_3 \cdot \text{FreqScore} + 
w_4 \cdot \text{DictScore}
\end{align}

donde:
\begin{itemize}
    \item $\text{Bigramas}_{\text{encontrados}}$ y $\text{Trigramas}_{\text{encontrados}}$ son los n-gramas del texto descifrado que coinciden con n-gramas comunes en español.
    \item $\text{FreqScore}$ evalúa si los n-gramas más frecuentes en el texto descifrado son también frecuentes en español.
    \item $\text{DictScore}$ mide el porcentaje de palabras descifradas que aparecen en un diccionario español.
    \item $w_1, w_2, w_3, w_4$ son pesos que determinan la importancia relativa de cada componente.
\end{itemize}

Esta función multicritério permite capturar diferentes aspectos de la coherencia lingüística, aumentando la robustez del algoritmo.
\end{cryptoanalysis}

\subsection{Resultados Experimentales}

\begin{securitygoodpractice}
\textbf{Caso de Estudio: Descifrado de un Texto Cifrado}

Al aplicar este enfoque a un texto cifrado mediante sustitución monoalfabética, se logró obtener un descifrado con un 86.90\% de coherencia:

\begin{quote}
ACTUALMENTE DISPONEMOS DE UNA GRAN CANTIDAD DE NAYEGADORES FEK PARA ELEGIR EL NUESTRO CONCEPTOS COMO LA SEGURIDAD O EL...
\end{quote}

El texto resultante es mayoritariamente legible, con solo algunos errores (como "NAYEGADORES FEK" en lugar de "NAVEGADORES WEB") que podrían corregirse manualmente o mediante un refinamiento adicional del algoritmo.

El tiempo de ejecución fue de aproximadamente 2 minutos con 20,000 iteraciones del algoritmo de optimización, lo que demuestra la eficiencia del enfoque incluso para textos de longitud considerable.
\end{securitygoodpractice}

\subsection{Consideraciones y Limitaciones}

\begin{securitywarning}
\textbf{Factores que Afectan el Rendimiento del Descifrado}

El rendimiento del algoritmo puede verse afectado por diversos factores:

\begin{itemize}
    \item \textbf{Longitud del texto}: Textos muy cortos pueden no proporcionar suficientes estadísticas para un descifrado confiable.
    \item \textbf{Caracteres especiales}: La presencia de símbolos, números o signos de puntuación puede complicar el análisis.
    \item \textbf{Modificaciones al cifrado básico}: Variantes como cifrados polialfabéticos o homofónicos requieren enfoques más sofisticados.
    \item \textbf{Especificidad del contenido}: Textos con terminología técnica o distribuciones atípicas de letras pueden resultar más difíciles de descifrar.
\end{itemize}

Para superar estas limitaciones, se pueden implementar estrategias como el análisis contextual, la incorporación de conocimiento del dominio y el refinamiento manual interactivo.
\end{securitywarning}

\section{Conclusiones}

La implementación de un decodificador automático para cifrados monoalfabéticos demuestra cómo los principios fundamentales del criptoanálisis de frecuencias, combinados con técnicas modernas de optimización combinatoria, pueden romper eficientemente esquemas de cifrado que alguna vez se consideraron seguros.

El éxito del enfoque descrito, con una coherencia del 86.90\%, ilustra por qué los cifrados de sustitución simple ya no se utilizan en aplicaciones de seguridad modernas. Al mismo tiempo, el estudio de estas técnicas proporciona una base sólida para comprender métodos criptográficos más avanzados y sus vulnerabilidades potenciales.

Este caso práctico refuerza un principio fundamental de la seguridad informática: la importancia de evaluar los sistemas criptográficos no solo desde la perspectiva de su diseño teórico, sino también considerando su resistencia a técnicas de análisis establecidas y emergentes.

\section{Descifrado de criptogramas}

\subsection{Séptimo mensaje - Datos}

Se nos proporciona la siguiente información:

\begin{itemize}
\item $n = 2581$
\item $p = 29$
\item $q = 89$
\item $d = 5$
\item $z = (p-1)(q-1) = (29-1)(89-1) = 2464$
\end{itemize}

Y el siguiente mensaje cifrado:

\begin{verbatim}
1236 2131 15 2131 202 2069 1035 2069 1104 662
\end{verbatim}

\subsection{Cálculo del inverso multiplicativo módulo $z$}

Necesitamos encontrar $e$ tal que $e \cdot d \equiv 1 \pmod{z}$, es decir, el inverso multiplicativo de $d$ módulo $z$. Para ello utilizaremos el algoritmo de Euclides extendido:
\\\\
\textbf{Inicialización del algoritmo de Euclides extendido:}

Estamos buscando $e$ tal que $e \cdot 5 \equiv 1 \pmod{2464}$

Inicializamos las variables según el algoritmo:

$g_0 = 2464$, $g_1 = 5$
$u_0 = 1$, $u_1 = 0$
$v_0 = 0$, $v_1 = 1$


\textbf{Iteración 1:}

Calculamos $y_2 = \lfloor g_0 / g_i \rfloor = \lfloor 2464 / 5 \rfloor = 492$

Calculamos $g_2 = g_0 - y_2 \cdot g_i = 2464 - 492 \cdot 5 = 4$

Calculamos $u_2 = u_0 - y_2 \cdot u_i = 1 - 492 \cdot 0 = 1$

Calculamos $v_2 = v_0 - y_2 \cdot v_i = 0 - 492 \cdot 1 = -492$



\textbf{Iteración 2:}

Calculamos $y_3 = \lfloor g_1 / g_i \rfloor = \lfloor 5 / 4 \rfloor = 1$

Calculamos $g_3 = g_1 - y_3 \cdot g_i = 5 - 1 \cdot 4 = 1$

Calculamos $u_3 = u_1 - y_3 \cdot u_i = 0 - 1 \cdot 1 = -1$

Calculamos $v_3 = v_1 - y_3 \cdot v_i = 1 - 1 \cdot -492 = 493$



\textbf{Iteración 3:}

Calculamos $y_4 = \lfloor g_2 / g_i \rfloor = \lfloor 4 / 1 \rfloor = 4$

Calculamos $g_4 = g_2 - y_4 \cdot g_i = 4 - 4 \cdot 1 = 0$

Calculamos $u_4 = u_2 - y_4 \cdot u_i = 1 - 4 \cdot -1 = 5$

Calculamos $v_4 = v_2 - y_4 \cdot v_i = -492 - 4 \cdot 493 = -2464$



\textbf{Resultado final:}

El algoritmo ha terminado porque $g_3 = 0$

El inverso se encuentra en el valor de $v$ en el penúltimo paso: $v_2 = 493$

Como $v_2 \geq 0$, no es necesario ajustar el valor.

Por lo tanto, el inverso multiplicativo de 5 módulo 2464 es 493

Verificación: $5 \cdot 493 \equiv 1 \pmod{2464}$

\begin{table}[h]
\centering
\begin{tabular}{|c|c|c|c|c|}
\hline
$i$ & $y_i$ & $g_i$ & $u_i$ & $v_i$ \\ \hline
0 & - & 2464 & 1 & 0 \\ \hline
1 & - & 5 & 0 & 1 \\ \hline
2 & 492 & 4 & 1 & -492 \\ \hline
3 & 1 & 1 & -1 & 493 \\ \hline
4 & 4 & 0 & 5 & -2464 \\ \hline
\end{tabular}
\caption{Cálculo del inverso modular utilizando el algoritmo de Euclides extendido}
\label{tab:euclides}
\end{table}

Por lo tanto, $e = 493$.

\subsection{Descifrado del mensaje}

Para descifrar el mensaje, utilizamos la fórmula $M = C^e \bmod n$, donde $e = 493$ y $n = 2581$.

\subsubsection{Proceso de descifrado}

$C = 1236$: $M = 1236^{493} \bmod 2581 = 84$ (ASCII: 'T')

$C = 2131$: $M = 2131^{493} \bmod 2581 = 69$ (ASCII: 'E')

$C = 15$: $M = 15^{493} \bmod 2581 = 76$ (ASCII: 'L')

$C = 2131$: $M = 2131^{493} \bmod 2581 = 69$ (ASCII: 'E')

$C = 202$: $M = 202^{493} \bmod 2581 = 86$ (ASCII: 'V')

$C = 2069$: $M = 2069^{493} \bmod 2581 = 73$ (ASCII: 'I')

$C = 1035$: $M = 1035^{493} \bmod 2581 = 83$ (ASCII: 'S')

$C = 2069$: $M = 2069^{493} \bmod 2581 = 73$ (ASCII: 'I')

$C = 1104$: $M = 1104^{493} \bmod 2581 = 79$ (ASCII: 'O')

$C = 662$: $M = 662^{493} \bmod 2581 = 78$ (ASCII: 'N')

\begin{table}[h]
\centering
\begin{tabular}{|c|c|c|}
\hline
$C$ & $M$ & ASCII \\ \hline
1236 & 84 & 'T' \\ \hline
2131 & 69 & 'E' \\ \hline
15 & 76 & 'L' \\ \hline
2131 & 69 & 'E' \\ \hline
202 & 86 & 'V' \\ \hline
2069 & 73 & 'I' \\ \hline
1035 & 83 & 'S' \\ \hline
2069 & 73 & 'I' \\ \hline
1104 & 79 & 'O' \\ \hline
662 & 78 & 'N' \\ \hline
\end{tabular}
\caption{Descifrado RSA: $m = c^e \bmod n$}
\label{tab:descifrado}
\end{table}
\subsubsection{Mensaje descifrado}

El mensaje descifrado es:

\begin{verbatim}
TELEVISION
\end{verbatim}


\section{Conclusiones}

El estudio detallado del algoritmo DES ha permitido comprender la estructura y funcionamiento de uno de los cifrados simétricos más influyentes en la historia de la criptografía. Aunque actualmente se considera obsoleto para aplicaciones de alta seguridad, sus principios de diseño continúan informando el desarrollo de algoritmos modernos.

La exploración de herramientas criptográficas del Criptolab ha demostrado la importancia de contar con recursos especializados tanto para fines educativos como para evaluación de seguridad. JCrypTool, CrypTool y DESCrack ofrecen perspectivas complementarias sobre la implementación, visualización y vulnerabilidades de diversos algoritmos criptográficos.

El análisis de frecuencias, aunque limitado en su aplicación a cifrados modernos, sigue siendo una técnica fundamental para comprender los principios básicos del criptoanálisis y para abordar ciertos tipos de cifrados históricos o básicos. Esta técnica ilustra la importancia de diseñar sistemas criptográficos que resistan al análisis estadístico.

Los hallazgos de este laboratorio refuerzan la máxima de que la seguridad criptográfica no debe depender del desconocimiento del algoritmo (principio de Kerckhoffs), sino de la robustez matemática del diseño y la gestión adecuada de claves. Asimismo, se evidencia la necesidad de evolución constante en el campo, dado que los avances en capacidad computacional y técnicas de ataque hacen que sistemas considerados seguros en el pasado se vuelvan vulnerables con el tiempo.

\begin{securitygoodpractice}
	Recomendaciones prácticas para la implementación de sistemas criptográficos:
	\begin{itemize}
		\item Utilizar algoritmos estandarizados y ampliamente revisados como AES en lugar de crear implementaciones propias.
		\item Mantener longitudes de clave adecuadas según el nivel de seguridad requerido (mínimo 128 bits para cifrado simétrico, 2048 bits para RSA).
		\item Implementar mecanismos robustos de gestión de claves, incluyendo generación, almacenamiento y renovación periódica.
		\item Combinar múltiples capas de seguridad, evitando depender exclusivamente de un único mecanismo criptográfico.
	\end{itemize}
\end{securitygoodpractice}

\section*{Agradecimientos}
Los autores agradecen al profesor Gustavo A. Isaza por su guía y apoyo durante el desarrollo de este laboratorio, así como al Departamento de Ingeniería en Sistemas y Computación de la Universidad de Caldas por proporcionar los recursos necesarios para la realización de estas pruebas.

\begin{thebibliography}{00}
	\bibitem{des_standard} National Bureau of Standards. ``Data Encryption Standard''. Federal Information Processing Standards Publication 46. 1977.

	\bibitem{schneier} Schneier, B. ``Applied Cryptography: Protocols, Algorithms, and Source Code in C''. John Wiley \& Sons, 2007.

	\bibitem{criptored} Criptored. ``Repositorio de Herramientas Criptográficas''. \url{http://www.criptored.upm.es/paginas/software.htm}, 2023.

	\bibitem{cryptool} CrypTool Project. ``CrypTool Documentation''. \url{https://www.cryptool.org/}, 2023.

	\bibitem{jcryptool} JCrypTool Team. ``JCrypTool - Eclipse-based Crypto Toolkit''. \url{https://www.cryptool.org/en/jct/}, 2023.

	\bibitem{analisis_frec} Singh, S. ``The Code Book: The Science of Secrecy from Ancient Egypt to Quantum Cryptography''. Anchor Books, 2000.

	\bibitem{idea} Lai, X., Massey, J. ``A Proposal for a New Block Encryption Standard''. Advances in Cryptology — EUROCRYPT '90. Lecture Notes in Computer Science, 1991.

	\bibitem{modern_crypto} Katz, J., Lindell, Y. ``Introduction to Modern Cryptography''. CRC Press, 2020.
\end{thebibliography}

% \begin{appendix}
% 	\section{Tablas Completas del Algoritmo DES}

% 	\subsection{Tablas de Permutación}
% 	\begin{table}[ht]
% 		\centering
% 		\caption{Tablas de Permutación Completas}
% 		\begin{tabular}{|c|c|c|c|c|c|c|c|}
% 			\hline
% 			\multicolumn{8}{|c|}{Permutación Inicial (IP)}      \\
% 			\hline
% 			58 & 50 & 42 & 34 & 26 & 18 & 10 & 2                \\
% 			60 & 52 & 44 & 36 & 28 & 20 & 12 & 4                \\
% 			62 & 54 & 46 & 38 & 30 & 22 & 14 & 6                \\
% 			64 & 56 & 48 & 40 & 32 & 24 & 16 & 8                \\
% 			57 & 49 & 41 & 33 & 25 & 17 & 9  & 1                \\
% 			59 & 51 & 43 & 35 & 27 & 19 & 11 & 3                \\
% 			61 & 53 & 45 & 37 & 29 & 21 & 13 & 5                \\
% 			63 & 55 & 47 & 39 & 31 & 23 & 15 & 7                \\
% 			\hline
% 			\multicolumn{8}{|c|}{Permutación Final (IP$^{-1}$)} \\
% 			\hline
% 			40 & 8  & 48 & 16 & 56 & 24 & 64 & 32               \\
% 			39 & 7  & 47 & 15 & 55 & 23 & 63 & 31               \\
% 			38 & 6  & 46 & 14 & 54 & 22 & 62 & 30               \\
% 			37 & 5  & 45 & 13 & 53 & 21 & 61 & 29               \\
% 			36 & 4  & 44 & 12 & 52 & 20 & 60 & 28               \\
% 			35 & 3  & 43 & 11 & 51 & 19 & 59 & 27               \\
% 			34 & 2  & 42 & 10 & 50 & 18 & 58 & 26               \\
% 			33 & 1  & 41 & 9  & 49 & 17 & 57 & 25               \\
% 			\hline
% 		\end{tabular}
% 	\end{table}

% 	\subsection{Tabla de Expansión E}
% 	\begin{table}[ht]
% 		\centering
% 		\caption{Tabla de Expansión E}
% 		\begin{tabular}{|c|c|c|c|c|c|}
% 			\hline
% 			32 & 1  & 2  & 3  & 4  & 5  \\
% 			4  & 5  & 6  & 7  & 8  & 9  \\
% 			8  & 9  & 10 & 11 & 12 & 13 \\
% 			12 & 13 & 14 & 15 & 16 & 17 \\
% 			16 & 17 & 18 & 19 & 20 & 21 \\
% 			20 & 21 & 22 & 23 & 24 & 25 \\
% 			24 & 25 & 26 & 27 & 28 & 29 \\
% 			28 & 29 & 30 & 31 & 32 & 1  \\
% 			\hline
% 		\end{tabular}
% 	\end{table}

% 	\subsection{Tablas S-Box Completas}
% 	\begin{table}[ht]
% 		\centering
% 		\caption{S-Box 1}
% 		\begin{tabular}{|c|c|c|c|c|c|c|c|c|c|c|c|c|c|c|c|c|}
% 			\hline
% 			\multicolumn{17}{|c|}{S1}                                                        \\
% 			\hline
% 			  & 0  & 1  & 2  & 3 & 4  & 5  & 6  & 7  & 8  & 9  & 10 & 11 & 12 & 13 & 14 & 15 \\
% 			\hline
% 			0 & 14 & 4  & 13 & 1 & 2  & 15 & 11 & 8  & 3  & 10 & 6  & 12 & 5  & 9  & 0  & 7  \\
% 			1 & 0  & 15 & 7  & 4 & 14 & 2  & 13 & 1  & 10 & 6  & 12 & 11 & 9  & 5  & 3  & 8  \\
% 			2 & 4  & 1  & 14 & 8 & 13 & 6  & 2  & 11 & 15 & 12 & 9  & 7  & 3  & 10 & 5  & 0  \\
% 			3 & 15 & 12 & 8  & 2 & 4  & 9  & 1  & 7  & 5  & 11 & 3  & 14 & 10 & 0  & 6  & 13 \\
% 			\hline
% 		\end{tabular}
% 	\end{table}

% 	\subsection{Tabla de Permutación P}
% 	\begin{table}[ht]
% 		\centering
% 		\caption{Tabla de Permutación P}
% 		\begin{tabular}{|c|c|c|c|c|c|c|c|}
% 			\hline
% 			16 & 7  & 20 & 21 & 29 & 12 & 28 & 17 \\
% 			1  & 15 & 23 & 26 & 5  & 18 & 31 & 10 \\
% 			2  & 8  & 24 & 14 & 32 & 27 & 3  & 9  \\
% 			19 & 13 & 30 & 6  & 22 & 11 & 4  & 25 \\
% 			\hline
% 		\end{tabular}
% 	\end{table}

% \end{appendix}
\end{document}

