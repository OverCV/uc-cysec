\documentclass[12pt,a4paper]{article}

% Paquetes esenciales para artículos de Ciberseguridad
\usepackage{cite}
\usepackage{amsmath,amssymb,amsfonts}
\usepackage{algorithm}
\usepackage{algpseudocode}
\usepackage{graphicx}
\usepackage{textcomp}
\usepackage{xcolor}
\usepackage{booktabs}
\usepackage{array}
\usepackage{multirow}
\usepackage{siunitx}
\usepackage{listings}

\usepackage{tikz}
\usetikzlibrary{shapes,arrows,positioning,fit,backgrounds,calc}
\usepackage{pgfplots}

\pgfplotsset{compat=1.16}

\usepackage{hyperref}
\usepackage{tabularx}
\usepackage{enumitem}
\usepackage{mdframed}
\usepackage[left=2.5cm,right=2.5cm,top=2.5cm,bottom=2.5cm]{geometry}
\usepackage{authblk} % Para gestionar autores y afiliaciones de manera sencilla

\usepackage[spanish]{babel} % idioma en español


% Definición de colores para ciberseguridad
\definecolor{alertred}{RGB}{235,54,42}
\definecolor{warningyellow}{RGB}{250,190,40}
\definecolor{securegreen}{RGB}{46,204,113}
\definecolor{infoblue}{RGB}{52,152,219}
\definecolor{vulnerablepurple}{RGB}{155,89,182}
\definecolor{codegray}{RGB}{240,240,240}
\definecolor{commentgreen}{RGB}{63,127,95}
\definecolor{codeblue}{RGB}{0,0,255}
\definecolor{codered}{RGB}{176,0,0}
\definecolor{cryptogreen}{RGB}{0,128,0}
\definecolor{encryptionbg}{RGB}{245,245,245}

% Configuración para listings con soporte mejorado para bash
\lstdefinelanguage{Bash}{
  keywords={if, then, else, elif, fi, for, do, done, while, until, case, esac, function},
  keywordstyle=\color{blue},
  sensitive=true,
  comment=[l]{\#},
  commentstyle=\color{commentgreen},
  stringstyle=\color{red!70!black},
  morestring=[b]",
  morestring=[b]'
}

% Definición para código fuente genérico incluyendo lenguajes de programación comunes
\lstdefinestyle{codestyle}{
    basicstyle=\ttfamily\small,
    keywordstyle=\color{codeblue},
    stringstyle=\color{codered},
    commentstyle=\color{commentgreen},
    showstringspaces=false,
    numbers=left,
    numberstyle=\tiny\color{gray},
    numbersep=5pt,
    frame=single,
    backgroundcolor=\color{codegray},
    breaklines=true,
    breakatwhitespace=true,
    tabsize=4,
    captionpos=b,
    literate=
        {\$}{{{\color{codeblue}\$}}}1
        {>}{{{\color{codered}\textgreater}}}1
        {<}{{{\color{codered}\textless}}}1
        {|}{{{\color{codered}|}}}1
        {&}{{{\color{codered}\&}}}1
}

% Definición para mostrar algoritmos criptográficos o pseudocódigo
\lstdefinestyle{cryptoalgo}{
    basicstyle=\ttfamily\small,
    keywordstyle=\color{cryptogreen}\bfseries,
    stringstyle=\color{codered},
    commentstyle=\color{commentgreen}\itshape,
    showstringspaces=false,
    numbers=left,
    numberstyle=\tiny\color{gray},
    numbersep=5pt,
    frame=single,
    backgroundcolor=\color{encryptionbg},
    breaklines=true,
    breakatwhitespace=true,
    tabsize=4,
    captionpos=b,
    emphstyle=\color{blue},
    emph={permutación, expansión, sustitución, rotación, desplazamiento, cifrado, descifrado, bloque, ronda, clave}
}

\lstset{style=codestyle}

% Estilos para bloques de alerta de seguridad
\newmdenv[
  linecolor=alertred,
  backgroundcolor=alertred!10,
  frametitle={Alerta de Seguridad},
  frametitlebackgroundcolor=alertred!40,
  frametitlefont=\bfseries\color{white},
  roundcorner=4pt
]{securityalert}

\newmdenv[
  linecolor=warningyellow,
  backgroundcolor=warningyellow!10,
  frametitle={Advertencia},
  frametitlebackgroundcolor=warningyellow!40,
  frametitlefont=\bfseries\color{black},
  roundcorner=4pt
]{securitywarning}

\newmdenv[
  linecolor=securegreen,
  backgroundcolor=securegreen!10,
  frametitle={Buena Práctica},
  frametitlebackgroundcolor=securegreen!40,
  frametitlefont=\bfseries\color{white},
  roundcorner=4pt
]{securitygoodpractice}

% Definición para métodos criptográficos
\newmdenv[
  linecolor=infoblue,
  backgroundcolor=infoblue!5,
  frametitle={Método Criptográfico},
  frametitlebackgroundcolor=infoblue!40,
  frametitlefont=\bfseries\color{white},
  roundcorner=4pt
]{cryptomethod}

% Definición para análisis criptográfico
\newmdenv[
  linecolor=vulnerablepurple,
  backgroundcolor=vulnerablepurple!5,
  frametitle={Análisis Criptográfico},
  frametitlebackgroundcolor=vulnerablepurple!40,
  frametitlefont=\bfseries\color{white},
  roundcorner=4pt
]{cryptoanalysis}

% Definición de BibTeX
\def\BibTeX{{\rm B\kern-.05em{\sc i\kern-.025em b}\kern-.08em
    T\kern-.1667em\lower.7ex\hbox{E}\kern-.125emX}}

\begin{document}

\title{\LARGE \textbf{Laboratorio: Protocolos Seguros y Seguridad Perimetral}}

\author{\textbf{Profesor Gustavo Adolfo Echeverri\\}}
\author[1]{Juan Carlos Charfuelan Caipe}
\author[2]{Fabian Alberto Guancha Vera}
\author[3]{Over Haider Castrillón Valencia}


\affil[1]{Departamento de Ingeniería en Sistemas y Computación, Universidad de Caldas}
\affil{Manizales, Colombia}
\date{\today}
\maketitle

\section*{Introducción}
% Aquí contextualizas objetivos y alcance del laboratorio.

\section{Configurar y desplegar un Web Server Seguro TLS/SSL}
\subsection{Mediante OpenSSL + Apache}
\subsubsection{Generación de certificados}
% Describe pasos de generación de certificados y configuración de Apache.

\subsubsection{Configuración de Apache}
% Describe pasos de configuración de Apache.

\subsection{Mediante código}
\subsubsection{Esquema de servidor Node.js}
% Esquema de servidor HTTPS en tu lenguaje preferido.

\subsubsection{Esquema de servidor Python}
% Esquema de servidor HTTPS en tu lenguaje preferido.

\subsubsection{Esquema de servidor Java}
% Esquema de servidor HTTPS en tu lenguaje preferido.



\section{Firma y verificación de documentos con PGP}
\subsection{Firma de un documento}
\subsubsection{Generación de clave}
% Uso de GnuPG CLI o librería Python.

\subsubsection{Firma con GnuPG}
% Ejemplo mínimo de firma.

\subsection{Verificación de firma}
\subsubsection{Proceso de verificación}
% Comandos o llamadas a la API de GnuPG.

\subsection{Cifrado y descifrado con la lib de GnuPG en Python}
\subsubsection{Ejemplo de cifrado}
% Ejemplo mínimo de cifrado.

\subsubsection{Ejemplo de descifrado}
% Ejemplo mínimo de descifrado.

\section{Entorno de filtrado de paquetes con IPTABLES / UFW}
\subsection{IPTABLES}
\subsubsection{Regla para Servicio A}
% Define reglas para al menos 3 servicios o protocolos.

\subsubsection{Regla para Servicio B}
% Define reglas para al menos 3 servicios o protocolos.

\subsection{UFW}
\subsubsection{Regla para Servicio C}
% Define reglas para al menos 3 servicios o protocolos.

\section{Despliegue de un IDS con Snort}
\subsection{Instalación de Snort}
% Instalación de Snort.

\subsection{Configuración de reglas}
\subsubsection{Ejemplo de regla de detección}
% Configuración de reglas para Snort.

\subsection{Prueba de detección}
\subsubsection{Simulación de ataque}
% Prueba de detección de Snort.

\section{Conclusiones}
% Reflexiones finales y aprendizajes.

\end{document}



sudo snort -dev -i eth0 -c \etc